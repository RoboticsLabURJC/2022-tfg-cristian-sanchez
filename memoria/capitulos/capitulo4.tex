\chapter{Diseño}
\label{cap:capitulo4}

Tras haber puesto en contexto todo lo anterior, en este capítulo se expondrá, de forma detallada, el proceso seguido para conseguir que un dron detecte y navegue hacia una señal \ac{RF}.\\

Además, se mostrará el desarrollo de una aplicación responsiva, que simula el comportamiento de una señal (en un espacio libre de obstáculos), basada en la aproximación de Friis.\\

Por último, se buscará determinar cuál de los métodos empleados es mejor y por qué, a través de diversas métricas comparativas que se expondrán en detalle posteriormente.\\

\section{Preparación del entorno}
\label{sec:preparacion_del_entorno}

Lo primero que se debía conseguir, era un entorno de simulación compatible con \ac{ROS}, así como un sistema de control de versiones, que nos permitiera mantener la trazabilidad y los backups a mano. Por ello, se estableció un repositorio común en GitHub y se usó el paquete de herramientas dispuesto por \textbf{JdeRobot}.

\subsection{JdeRobot - drones}
\label{subsec:jderobot_drones}

Gracias a esta plataforma, se pudo obtener los modelos y los módulos necesarios para poder simular en Gazebo, el desempeño de un cuadracóptero provisto de un sistema de autopilot PX4.\\

El modelo usado es el \textbf{3DR Iris simulado}, con un plugin de una cámara frontal. Este dispositivo utiliza MAVROS, para realizar la comunicación, lo que nos permite enviar y recibir mesajes ROS compatibles con el protocolo de comunicaciones típico de estas aeronaves, MAVLink.\\

\subsection{Teleoperador}
\label{subsec:teleoperador}

ToDo...
