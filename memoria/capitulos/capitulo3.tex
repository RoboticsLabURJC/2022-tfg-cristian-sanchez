\chapter{Plataformas de desarrollo y herramientas utilizadas}
\label{cap:capitulo3}

Para hacer realidad todo lo mencionado en capítulos anteriores, se usaron diversos recursos ingenieriles que hicieron posible el proyecto. A continuación, se detallará cada uno de ellos.

\section{Lenguajes de programación}
\label{sec:lenguajes_programacion}

\subsection{Python}
\label{subsec:python}

A día de hoy, considerado el lenguaje de programación más popular, se ideó en 1991 por Guido van Rossum y se desarrolló en la Python Software Foundation. Es interpretado, es decir, usa un programa que traduce las líneas de código para la máquina en tiempo de ejecución (lo cual lo hace más intuitivo pero menos eficiente). Además, permite la programación orientada a objetos en alto nivel, lo que ofrece gran dinamismo a la hora de usarlo.\\

Debido a su amplia popularidad, podemos acceder a una gran variedad de módulos y utilidades desarrollados por la comunidad, los cuales se integran perfectamente en la resolución de nuestro problema. Entraremos en más detalles en apartados posteriores.

\subsection{C++}
\label{subsec:cplusplus}

Seguido muy de cerca en fama, se encuentra el lenguaje de programación creado por Bjarne Stroustrup, en los laboratorios Bell en 1971. En este caso es compilado, lo que implica la traducción y enlazado previo a la ejecución. De corte más eficiente que Python, también permite la programación orientada a objetos. Se sitúa a medio camino entre un lenguaje de alto nivel y uno de bajo nivel.\\

\section{\ac{ROS}}
\label{sec:ros}

Si se habla de robótica, se habla de \ac{ROS}, ya que es el medio predilecto para el desarrollo de soluciones de este ámbito, pero, ¿qué es exactamente \ac{ROS}?.\\

Se trata de un \emph{middleware}, es decir, una infraestructura software situada entre el sistema operativo y el desarrollador, que incluye una serie de módulos y funcionalidades enfocadas al desarrollo de aplicaciones robóticas. La idea detrás, busca estandarizar soluciones que no dependan de los drivers de cada sensor y actuador presentes. De forma general, se trata de una arquitectura basada en nodos que se comunican entre sí, transmitiendo una serie de mensajes propios, a través de canales compartidos llamados \emph{topics}.\\

Entre las herramientas usadas en este proyecto, se encuentran las siguientes.

\subsection{Gazebo 11}
\label{subsec:gazebo}

ToDo...

\subsection{Rviz}
\label{subsec:rviz}

ToDo...

\section{Plataformas de programación}
\label{sec:plataformas_de_programacion}

\subsection{Visual Studio Code}
\label{subsec:visual_studio_code}

ToDo...

\subsection{Github}
\label{subsec:github}

ToDo...

\section{Módulos}
\label{sec:modulos}

ToDo...

\subsection{OpenCV}
\label{subsec:opencv}

ToDo...

\subsection{Matplotlib}
\label{subsec:matplotlib}

ToDo...

\subsection{MavROS}
\label{subsec:mavros}

ToDo...

\subsection{PX4}
\label{subsec:px4}

ToDo...

\subsection{jderobot}
\label{subsec:jderobot}

ToDo...