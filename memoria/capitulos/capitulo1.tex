\chapter{Introducción}
\label{cap:capitulo1}
\setcounter{page}{1}

En la actualidad, la tecnología forma parte de nuestro día a día. Prácticamente, constituye un elemento imprescindible para llevar a cabo cualquier actividad, sea profesional o cotidiana. ¿Su función? solucionar problemas para hacernos la vida más sencilla.\\

Con esto en mente, se presenta la robótica, pero \textbf{¿qué es la robótica?}. Según la \ac{RAE}, la robótica se define cómo \emph{``técnica que aplica la informática al diseño y empleo de aparatos que, en sustitución de personas, realizan operaciones o trabajos, por lo general en instalaciones industriales.''} (Real Academia Española, s.f., definición 2) \cite{rae-robotica}. Sin embargo, no es precisa, por ello una definición más concreta podría ser, ciencia que engloba diversas ramas tecnológicas, encargada del estudio y diseño de dispositivos mecánicos, provistos de sensores y actuadores, capaces de realizar tareas a través de la extracción y posterior procesamiento de la información, con el fin de generar respuestas adecuadas para resolver determinados problemas \cite{revista-de-robots}.\\

Dentro de la robótica, existen diversas maneras de clasificar, sin embargo, una de las más comunes esta relacionada con la movilidad del dispositivo, esto es, si el mecanismo se puede desplazar por su entorno o no, por tanto se distingue lo siguiente:

\begin{enumerate}
	\item \textbf{Robótica industrial}: que involucra mecanismos fijos, capaces de realizar tareas de manera rápida, precisa y eficiente. Como es el caso de los brazos robóticos \cite{industrial-robot}.

	\item \textbf{Robótica móvil}: la cual abarca a los dispositivos móviles que se engloban en múltiples entornos y aplicaciones, como pueden ser, robótica aérea, terrestre y submarina \cite{mobile-robot}.
\end{enumerate}

\begin{figure} [h]
	\centering
	\subfigure[Robot industrial ABB]{\includegraphics[height=3cm]{imagenes/cap1/1_industrial_robot.jpeg}}
	\quad
	\subfigure[Robot móvil Spot]{\includegraphics[height=3cm]{imagenes/cap1/2_mobile_robot.jpeg}}
	\quad
	\subfigure[Robot móvil aéreo]{\includegraphics[height=3cm]{imagenes/cap1/1-5_real_drone.jpeg}}
	\caption{Robótica industrial VS robótica móvil}
	\label{fig:industrial_vs_mobile}
\end{figure}

Como tal, la robótica ayuda a resolver tareas repetitivas, peligrosas, delicadas y en ambientes problemáticos (conocidas como las 4 D's, \emph{dull, dirty, dangerous and dear}) \cite{4-d}. Sin embargo, uno de los problemas más complicados de abordar, es el \textbf{contexto}, es decir, la capacidad de entender y adaptarse a las circunstancias del problema, como por ejemplo en el caso de la conducción autónoma, donde detectar un simple peatón, puede derivar en infinitos inconvenientes (condiciones de visibilidad, clima, atuendo, entre muchos otros). Es ahí, donde se presenta el segundo gran protagonista, la \ac{IA} \cite{dworakowski2020robots}.\\

A continuación se definirán conceptos básicos como: ¿qué es exactamente un robot? ¿qué tipo de robot usaremos? ¿en qué consiste la \ac{IA} y que emplearemos? ¿cómo funcionan las señales que rastrearemos? entre otras cuestiones.

\section{Robots}
\label{sec:robots}

Un robot es un dispositivo provisto con \textbf{sensores}, o elementos capaces de extraer información del entorno (por ejemplo una cámara), \textbf{actuadores}, o elementos que permiten al dispositivo realizar acciones (por ejemplo un motor), y una \textbf{unidad de procesamiento}, que se encarga de generar acciones a través de la información obtenida con los sensores, todo ello mediante algoritmos \cite{Wang2022}.\\

\begin{figure} [H]
	\begin{center}
	\includegraphics[height=5cm]{imagenes/cap1/3_robot.png}
	\end{center}
	\caption[Definición de robot]{Definición de robot}
	\label{fig:robot_def}
\end{figure}

Según el problema que se quiera resolver conviene usar unos u otros. En nuestro caso buscamos un robot con capacidad de navegar, preferiblemente grandes distancias y que pueda tomar medidas de la intensidad de una señal de forma autónoma.\\

Existen múltiples robots capaces de satisfacer estas condiciones, vease, los AGV/AMR o plataformas robóticas terrestres ampliamente empleadas logística \cite{agv-vs-amr}; robots bipedos, los cuales emulan el movimiento humanoide, lo que aumenta su adaptabilidad a cualquier entorno real (ya que el mundo esta diseñado para la biomecánica humana), sin embargo, es bastante complejo replicar la marcha bípeda \cite{10.3389/fmech.2020.00011}; y por último, drones, las cuales detallaremos a continuación.

\subsection{Drones}
\label{subsec:drones}

Los \ac{UAS} tienen origen en la primera guerra mundial, con el biplano llamado \textbf{Kettering bug}. Se trataba de un torpedo que era lanzado desde una carretilla, capaz de volar de forma no tripulada, hasta que se liberaba de sus alas y caía sobre el objetivo \cite{kettering-bug}. Más tarde, entre la primera y segunda guerra mundial (1935), se diseño el \textbf{Queen Bee}, de donde surgió el termino \emph{``drone''}, como abeja macho en busca de la reina, que se trataba de un avión no tripulado, con el fin de servir de objetivo para realizar prácticas de artillería aérea \cite{queen-bee}. Sin embargo, no fue hasta \textbf{Operation Aphrodite}, en la segunda guerra mundial, donde realmente se vió el primer dron radio tripulado, con el fin de poder volar en entornos ``sucios'' o dirty, dado el nuevo paradigma de las bombas atómicas \cite{operation-aphrodite}.\\

\begin{figure} [h]
	\centering
	\subfigure[Kettering Bug]{\includegraphics[height=2.5cm]{imagenes/cap1/4_kettering_bug.jpeg}}
	\quad
	\subfigure[Queen Bee]{\includegraphics[height=2.5cm]{imagenes/cap1/5_queen_bee.jpeg}}
	\quad
	\subfigure[Aphrodite Operation]{\includegraphics[height=2.5cm]{imagenes/cap1/6_aphrodite.jpeg}}
	\caption{Historia de los drones}
	\label{fig:drone_history}
\end{figure}

Existen múltiples avances y ejemplos posteriores, pero en la actualidad podemos definir un \textbf{\ac{UAS}} teniendo en cuenta lo siguiente:

\begin{enumerate}
	\item \textbf{\ac{GCS}}: es la estación de tierra o el elemento encargado de controlar la nave.

	\item \textbf{Comunicación}: conecta y gestiona la transmisión de datos entre el \ac{UAV} y la \ac{GCS}, mediante \textbf{data links}, o canales de transmisión.
	
    \item \textbf{\ac{UAV}}: hace referencia directamente a la aeronave.
\end{enumerate} \cite{uas-definicion} \cite{gcs-definicion} \cite{data-link-definicion} \\

\begin{figure} [H]
	\begin{center}
	\includegraphics[height=6cm]{imagenes/cap1/7_drone_components.jpeg}
	\end{center}
	\caption[UAS]{UAS}
	\label{fig:drone_components}
\end{figure}

También cabe destacar que hay variedad de drones, según su peso y capacidad de carga de pago, o elementos que sea capaz de cargar, lo cual influye en la \textbf{legislación} detras de su uso (de forma general, cuanto mayor sea el peso, más legislación debe cumplir y mayores restricciones de uso tiene) \cite{drone-regulation}. Por ello, el dispositivo seleccionado es de la categoría más inferior o tambien denominados \ac{SUAV}.\\

Tal y como fue mencionado, la gran ventaja del uso de vehículos aéreos es poder evitar las irregularidades del terreno, sin embargo, hay ligados al uso de estos dispositivos ciertos problemas, como son el clima, la carga de pago que afecta a la autonomía (peso de las baterías), los interiores (afectan a la señal GPS), entre otros. Esto, se debe tener en cuenta de cara a la resolución del problema.\\

Agrupando la robótica y los drones, se pueden observar múltiples ejemplos de uso, uno muy conocido es el de un dron \emph{``sigue-persona''}, el cual permite a un \ac{SUAV} detectar y moverse al son de un objetivo móvil, tal y como puede ser una persona. Todo esto en síntesis con técnicas de \ac{IA}, la cual introduciremos a continuación.

\section{Inteligencia artificial}
\label{subsec:inteligencia_artificial}

La \ac{IA} ha tenido un auge importante en los últimos años, especialmente en el ámbito de la robótica dado su amplio abanico de soluciones sinérgicas con la misma, sin embargo, conviene conocer lo origenes. Quizás, la primera pregunta que se buscó responder fue la siguiente \textbf{¿Puede una máquina pensar?}, formulada en \emph{``Computing Machinery and Intelligence''} (Alan Turing, 1950), de donde surgió el famoso test de Turing, entre otras ideas \cite{turing-paper}. La búsqueda de la \ac{IA} enfrentó desafíos iniciales debido a la incapacidad de las primeras computadoras para almacenar datos y su elevado precio. Sin embargo, en 1956, se presentó el primer programa de \ac{IA} llamado \textbf{Logic Theorist} en el \textbf{Dartmouth Summer Research Project on Artificial Intelligence} \cite{logic-theorist}. Con el tiempo, la IA progresó con mejores algoritmos y mejoras en la capacidad de las computadoras. A pesar de esto, lograr los objetivos finales de la IA, como comprender el lenguaje humano y el pensamiento abstracto, sigue siendo un desafío a día de hoy \cite{history-ai}.\\

En general, la \ac{IA} es capaz de abordar los siguientes problemas:

\begin{enumerate}
	\item \textbf{Regresión}: se trata de aprendizaje supervisado, es decir, para construir el modelo, se emplea un conjunto de datos del que se conocen las salidas cómo las entradas a las que pertenecen. La idea es conseguir obtener una salida precisa dada una entrada concreta. Por ejemplo, dados los metros cuadrados de una vivienda (entrada), obtener su precio estimado (salida).

	\item \textbf{Clasificación}: se encarga de distribuir un conjunto de datos en sets con características comunes. Un ejemplo común es la segmentación de imagenes, donde se clasifica cada elemento de la imagen según su naturaleza.
	
    \item \textbf{Aprendizaje}: resuelve un problema a base de prueba y error, mediante un sistema de recompensas. Como por ejemplo \emph{``Stockfish''}, que es un modelo entrenado para ganar una partida de ajedrez en el menor número de movimientos posible, superando incluso a grandes maestros de la actualidad.
\end{enumerate} \cite{reg-class} \cite{chess}

\begin{figure} [H]
	\begin{center}
	\includegraphics[height=5.5cm]{imagenes/cap1/8_AI_types.png}
	\end{center}
	\caption[Clasificación de aprendizaje máquina]{Clasificación de aprendizaje máquina}
	\label{fig:ai_types}
\end{figure}

\subsection{Aprendizaje por refuerzo}
\label{subsec:aprendizaje_por_refuerzo}

Tal y como se comentó previamente, el aprendizaje por refuerzo, se basa en un sistema de \textbf{recompensas y penalizaciones}, que permite entrenar a un modelo para tomar buenas decisiones. Este enfoque se basa en los llamados \textbf{procesos de Markov}, que se definen como aquellos que, para un instante dado, contienen toda la información relevante sin depender de todos los procesos anteriores. \cite{learn}\\

En particular, hablamos de \textbf{agente}, o modelo encargado de tomar decisiones en un \textbf{entorno} (que es el medio en el que interactúa dicho agente, y está regido por una serie de reglas); \textbf{estados}, o circunstancias en la que se sitúa el agente en un determinado instante temporal; y \textbf{acciones}, o decisiones que toma el agente y que le permiten cambiar de estado. En términos de Markov, decimos que el estado actual no depende de todos los estados previos. \cite{props-learn}\\

\begin{figure} [H]
	\begin{center}
	\includegraphics[height=4cm]{imagenes/cap1/9_reinforcement.png}
	\end{center}
	\caption[Aprendizaje por refuerzo]{Aprendizaje por refuerzo}
	\label{fig:reinforcement_learning}
\end{figure}

Cabe destacar que, este enfoque está directamente extraido de la \textbf{psicología} y el estudio del comportamiento, donde en función de recompensas y castigos se induce al aprendizaje en distintas tareas, como por ejemplo, enseñar a jugar al ping pong a dos palomas \cite{psicologia-aprendizaje} \cite{skinner}.\\

Concretamente, se hará uso de la metodología \textbf{Q-Learning}, que busca generar una tabla numérica donde cada fila se interprete como un estado del robot, que puede ser su posición; y cada columna sea una determinada acción, como puede ser moverse hacia algún lugar. De este modo, y una vez construida la tabla (que se rellena con números en función de las recompensas y penalizaciones), el robot solo tiene que identificar en que estado se encuentra y elegir la columna con mayor valor numérico, lo que se traducirá en la mejor acción para dicho estado. \cite{q-learning}\\

Existen múltiples ejemplos de aplicación de esta metodología a casos reales, vease para controlar de forma adaptativa una señal de tráfico; para jugar a la Atari 2600; o para realizar un control híbrido sobre la navegación de un robot, entre otros. \cite{q-learning-app}\\

\section{Vigilancia del espectro electromagnético}
\label{subsec:señales}

Es el último elemento presente en el problema, que es la fuente de datos que se empleará para construir la inteligencia del robot. En este caso, hablamos de las señales, concretamente de señales \textbf{\ac{RF}}, como son por ejemplo Wi-Fi, radio FM, 4G, 5G, entre otros tipos de señales distribuidas a lo largo del espectro electromagnético.\\

Dicho espectro se divide por bandas de frecuencia, que se reparten para diversos uso. El ejemplo más claro es la banda FM de radio, que se reparte entre los 87-108 Mhz para España, donde cada emisora tiene un ancho asignado para emitir. \cite{bandw}\\

De este modo, se pueden encontrar soluciones a problemas como el rastreo de una señal de móvil para una persona perdida en la montaña, o seguir emisores concretos, como pueden ser convoys, entre otros. Lo único que hay que establecer, es la banda de frecuencia adecuada y establecer un dispositivo capaz de navegar hasta ella de manera autónoma, es por esto que la robótica es adecuada para resolver cuestiones de esta naturaleza.\\

Más adelante, se detallarán los parámetros relevantes de cara a trabajar con señales.

\begin{figure} [H]
	\begin{center}
	\includegraphics[height=4.5cm]{imagenes/cap1/10_friis.png}
	\end{center}
	\caption[Modelo de Friis]{Modelo de Friis}
	\label{fig:friis}
\end{figure}

\section{Síntesis}
\label{subsec:sintesis}

Para finalizar, cabe destacar algunos proyectos relacionados, donde se puede ver el uso conjunto de drones y robótica para la creación de un módulo ROS, en el que se engloban las comunicaciones para programar un sigue-persona \cite{tfm-pedro}, junto con el uso de estas aeronaves en síntesis con la \ac{IA}, para lograr la navegación en interiores de la misma \cite{paper-ia-dron}\\

En nuestro caso, se usará el mencionado dispositivo del tipo \ac{SUAV}, provisto de un receptor \ac{RF} como sensor, y los motores propios como actuadores. Se le agregarán algoritmos sistemáticos para compararlos con soluciones Q-Learning, en la tarea de resolver la detección y posterior navegación al origen de una señal \ac{RF}.