\chapter*{Anexo}
\label{cap:anexo}

A continuación muestro una serie de conceptos básicos relacionados con el estudio del procesamiento de señales\footnote[1]{Información extraida de la siguiente colección de videos didácticos \url{https://www.youtube.com/playlist?list=PL8bSwVy8_IcPCsBE71CYBLbQSS8ckWm6x}}:

\begin{enumerate}
	\item \emph{Señal}: se trata de una función que describe un fenómeno físico, y que se emplea para la transmisión de información.
    \item \emph{Dominio temporal}: establece el eje de abcisas con el tiempo.
    \item \emph{Dominio de la señal}: determina si la señal se expresa en tiempo o en frecuencia (a través de transformadas).
    \item \ac{ADC}: elemento electrónico que permite la conversión de señales analógicas a señales digitales.
    \item \ac{RSSI}: permite establecer el nivel de potencia de una señal, con respecto a 1 mW de potencia. Se expresa en dBm.
    \item \ac{SNR}: métrica que permite medir la potencia de la señal con respecto al ruido ambiente.
    \item \emph{Frecuencia}: parámetro de la función que define a la señal, el cual determina el número de veces que se repite en un segundo. Se mide en hercios (Hz).
    \item \emph{TX}: Se refiere a la transmisión de la señal.
    \item \emph{RX}: Hace referencia a la recepción de la señal.
\end{enumerate}